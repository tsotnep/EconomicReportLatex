\documentclass{article}
\usepackage[utf8]{inputenc}

\title{empiric}
\author{ }
\date{January 2016}

\usepackage{natbib}
\usepackage{graphicx}

\begin{document}

\maketitle

\section{Introduction}

Strengths and weaknesses of the paper of G.D. Abate

The paper of G.D. Abate is about the link between volatility and growth: a spatial econometrics approach presents the smooth analyses of countries volatility and growth in connection with the neighboring countries. He presents the analyses based on the Ramey\&Ramey standard model and also provides with some empirical results to illustrate and support his opinion.

I would like to present the strengths and then weaknesses of this work from my point of view.

\begin{itemize}

\item To begin with, the main advantage of the paper is that the author presents all the analyses in a brief way and also gives some new topics for future investigation and research. This is considerably meaningful advantage because it points out the fact that paper presents innovative ideas. Also, the author even makes an estimations of a new model SDM Ramey\&Ramey model with interaction effects.

\item Furthermore, he presents the data with full information and additionally he gives the sources %or references?
to access the data. In my opinion, this is a very useful information that is given to the reader as an opportunity of checking and over-viewing data that was used in the work.

\item Moreover, when Abate interprets direct and indirect aspects of the model he gives the recent information about the financial crisis that was propagated from USA. I think this connection with today's world makes the paper more more interesting.

\item One more advantage of the paper is that while showing the estimating procedures author is using bilateral trade weight matrix, instead of binary contiguity or inverse distance matrices that are more widely used in the literature and research papers. Furthermore, he gives us the reason and good explanation of selecting this matrix: �The motivation in considering bilateral trade weight matrix instead of the conventional geographical weight matrix comes from the fact that spatial weight matrices based on trade intensities are more appropriate in capturing economic spillovers than the geographical distance weight matrices. The use of bilateral trade weight matrix as a measure of economic linkages among countries is based on the economic theory which suggests that the existence of cross-border trade supports the prediction that economic outcomes across nations is not independent.�

\item Apart from this he also gives cogent reason what kind of problem the choice of matrix  may give - �One of the potential problems associated with using a spatial weight matrix constructed from bilateral trade is that volume of bilateral trade and growth may be determined simultaneously in the long-run equilibrium resulting in endogeneity problem�.

\item Moreover, he presents different weight matrix to show sensitivity of the results and after analyzing results he presents evidence why he used the bilateral weight matrix.
All the above listed reasons (especially reasons for weight matrix) show that the author expresses his own ideas and opinion on the topics he discusses in the paper and additionally, he supplies extra literature and comparisons with them that makes his work valuable.

\item The author also backs up his ideas with theoretical explanations, such as for example, why should, country specific and time period fixed effects, be included in the model and so on. This gives extra information to the reader and helps in understanding what author is doing and why. When he tries the estimation of this model he tells that new identification is needy for a variable that will affect output volatility both across country and over time. Next, he gives information that Ramey-Ramey identified government spending as a source of volatility, from my point of view he could have used new source of volatility and then compared gained results with the original estimations. But he points out that the motivation of including the government spending is that in the model in case of the measure of government-spending volatility might capture some effects on growth. Maybe this gives sufficient evidence to be used the same source of volatility.

\end{itemize}

And on the other hand: 

\begin{itemize}

\item I would like to underline some facts that could be improved in the paper. In the beginning, when Abate starts presenting the Ramey-Ramey Model he does not provide short and brief explanation of the model itself. Additionally, he presents Spatial Durbin Ramey-Ramey Model and in this case too, he does not present the SDM(Spatial Durbin Model) model itself. Also, he presents LM(Lagrange Multiplier) test without introducing the hypotheses. In my opinion paper has to be understandable as for professionals in this field as well as non-professionals - without reading large amount of indirectly related materials.

\item One more weakness that I do consider to be important is an interpretation of the results in the beginning of the paper when Abate presents non-spatial Ramey-Ramey Model. In the later sections he gives really nice and brief analyses of the attained results but for the first model it is missing. 

\item When he presents the variables of the model, he gives information that one of the variable - human capital � will be measured in the same way as Ramey-Ramey model. I think he could have suggested new estimation too.

\end{itemize}

In the end I can say that Abate is presenting the analyses based on the comparison with standard Ramey-Ramey Model, and after estimation each model he makes comparison the consistency is attained results with the original - old version of the model. I think this way is obviously considerable advantage of the paper. Moreover, if he uses some information from different resources always provides references to them and this makes easier for the interested readers to obtain more information and be convinced how the results come from.

In conclusion, I can say that the paper of Abate gives thorough analyses, and provides sufficient evidence of his work and steps. The main advantage is that the theoretical extension and the empirical finding obtained in this paper is a new direction in investigating the link between volatility and growth. Previous papers mainly focus on addition of some variables in the standard Ramey-Ramey model neglecting possible spatial interactions between countries.


\end{document}
