\documentclass{article}
\usepackage[utf8]{inputenc}

\title{empiric}
\author{ }
\date{January 2016}

\usepackage{natbib}
\usepackage{graphicx}

\begin{document}

\maketitle

\section{Introduction}

Strengths and weaknesses of the paper of G.D. Abate

The paper of G.D. Abate is about the link between volatility and growth: a spatial econometrics approach presents the smooth analyses of countries volatility and growth in connection with the neighboring countries. He presents the analyses based on the Ramey\&Ramey standard model and also provides with some empirical results to illustrate and support his opinion.\\
I would like to express my opinion with respect to this paper and present the strengths and weaknesses of this work from my point of view in this paper.
\\
To begin with, the main advantage of the paper is that the author presents all the analyses in a brief way and also gives some new topics for future investigation and research. This is considerably meaningful advantage because it points out the fact that paper presents innovative ideas. Also, even the author estimates a new model SDM Ramey\&Ramey model, model with interaction effects.
\\
Furthermore, he presents the data with full information and additionally he gives the sources to access the data. In my mind, this is significantly useful information that is given to the reader as an opportunity of checking and overviewing data that was presented in the work.
\\
Moreover, when Abate interprets direct and indirect of the model he gives the recent information about the financial crisis that was propagated from USA. I think this connection with todays� world gives more attraction to the paper.

One more advantage of the paper is that in order to show the estimation procedures author is using bilateral trade weight matrix, instead of binary contiguity or inverse distance matrices that are more commonly used in the literature and research works. Furthermore, he gives us the reason and good explanation of selecting this matrix: �The motivation in considering bilateral trade weight matrix instead of the conventional geographical weight matrix comes from the fact that spatial weight matrices based on trade intensities are more appropriate in capturing economic spillovers than the geographical distance weight matrices. 
The use of bilateral trade weight matrix as a measure of economic linkages among countries is based on the economic theory which suggests that the existence of cross-border trade supports the prediction that economic outcomes across nations is not independent.�

Apart from this he also gives cogent reason what kind of problem this choice of matrix  may cause - �One of the potential problems associated with using a spatial weight matrix constructed from bilateral trade is that volume of bilateral trade and growth may be determined simultaneously in the long-run equilibrium resulting in endogeneity problem�.

Moreover, he presents different weight matrix to show sensitivity of the results and after analyzing results he presents evidence why he used the bilateral weight matrix.
All the above listed reasons (especially reasons for weight matrix) show that the author shows his own ideas and opinion on the topics he discusses in the paper and additionally, he supplies extra literature and comparisions with them that makes his work interesting.

The author also backs up his ideas with theoretical explanations, such as for example, why should, country specific and time period fixed effects, be included in the model and so on. This gives extra information to the reader and helps in understanding what author is doing and why. When he tries the estimation of this model he tells that new identification is needy for a variable that will affect output volatility both across country and over time. Next, he gives information that Ramey-Ramey identified government spending as a source of volatility, from my point of view he could have use new source of volatility and then compare gained results with the original estimations. But he points out the motivation of include the government spending is that in the model in case the measure of government-spending volatility is capturing some effects on growth. Maybe this gives sufficient evidence to be used the same source of volatility.

I would like to underline some facts that could be improved in the paper. In the beginning, when Abate starts presenting the Ramey-Ramey Model he does not provides short and brief explanation of the model itself. Additionally, he presents Spatial Durbin Ramey-Ramey Model and in this case to he does not presents the SDM(Spatial Durbin Model) model itself too. Also, he presents LM(Lagrange Multiplier) test without introducing the hypotheses. Paper has to be understandable as for professionals in this field as well as non-professionals. 

One more weakness that I do consider is an interpretation of the results in the beginning of the paper when Abate presents non-spatial Ramey-Ramey Model. In the later sections he gives really nice and brief analyses of the attained results but for the first model it is missing. 

When he presents the variables of the model, he gives information that one of the variable - human capital � will be measured in the same way as Ramey-Ramey model. I think he could have suggested new estimation too.

In the end I can say that Abate is presenting the analyses based on the comparison with standard Ramey-Ramey Model, and after estimation each model he makes comparison the consistency of attained results with the original, old version of the model. I think this way is obviously considerable advantage of the paper. Moreover, if he uses some information from different resources always provides with them and makes easier for the readers to obtain more information and be convinced how the results come from.

In conclusion, I can say that the paper of Abate gives nice analyses, gives sufficient evidence of his work and steps. The main advantage is that the theoretical extension and the empirical finding obtained in this paper is a new direction in investigating the link between volatility and growth. Previous papers mainly focus on addition of some variables in the standard Ramey-Ramey model neglecting possible spatial interactions between countries.


\end{document}
